\section{Conclusion}

We presented SceneGuard, a training-time voice protection method based on scene-consistent audible background noise. Unlike existing defenses that rely on imperceptible perturbations, SceneGuard leverages naturally occurring acoustic scenes to create protective noise that is contextually appropriate and robust to common audio preprocessing operations.

Our experimental evaluation demonstrates that SceneGuard achieves strong protection against training attacks, degrading speaker similarity by 5.5\% with extremely high statistical significance ($p < 10^{-15}$, Cohen's $d = 2.18$). Critically, SceneGuard preserves speech usability, maintaining 98.6\% intelligibility (STOI = 0.986) and achieving low word error rate (3.6\%). Robustness evaluation shows that SceneGuard maintains or enhances protection under five common countermeasures, including MP3 compression, denoising, filtering, and downsampling.

These results suggest that audible, scene-consistent noise provides a practical alternative to imperceptible perturbations for voice protection. By deliberately using audible but natural noise, SceneGuard achieves robustness properties that imperceptible methods cannot match, addressing a fundamental limitation of existing approaches.

\subsection{Future Work}

Several directions merit further investigation:

\textbf{Multi-Speaker Evaluation}: Expanding evaluation to diverse speakers would strengthen generalization claims and enable analysis of protection effectiveness across vocal characteristics. Large-scale studies with hundreds of speakers would provide more comprehensive evidence.

\textbf{Adaptive SNR Selection}: Rather than using a fixed SNR range, future work could develop adaptive methods that select SNR based on speech content, acoustic conditions, or user preferences. This could optimize the protection-quality trade-off on a per-utterance basis.

\textbf{User Studies}: Subjective evaluation through listening tests would complement our objective metrics. Understanding user perception and acceptance of scene-consistent noise is critical for real-world deployment. Studies should assess whether users find protected speech natural and acceptable for their applications.

\textbf{Learned Scene Classification}: Our current implementation uses heuristic scene assignment for reproducibility. Incorporating learned ASC models could provide more accurate scene matching, potentially improving both protection and perceptual naturalness.

\textbf{Defense Against Adaptive Attacks}: Investigating potential adaptive attacks and developing corresponding defenses would strengthen SceneGuard's security guarantees. This includes studying scene-aware source separation and designing countermeasures.

\textbf{Real-World Deployment}: Implementing SceneGuard in practical systems such as voice assistants or communication apps would enable evaluation under realistic usage conditions. This includes studying computational efficiency, user experience, and protection effectiveness in diverse deployment scenarios.

By addressing the fundamental fragility of imperceptible perturbations, SceneGuard opens new directions for robust voice protection research. We hope our work inspires further investigation into audible but natural defense mechanisms for speech and audio privacy.

